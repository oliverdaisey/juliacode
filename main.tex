\documentclass[12pt,reqno,a4paper]{amsart}
\usepackage[doi=true,isbn=false,style=alphabetic,sorting=nyt,backend=biber,maxnames=5,maxalphanames=5,giveninits=true]{biblatex}
\AtEveryBibitem{\clearlist{language}}
\bibliography{main}

\usepackage[breaklinks,colorlinks,plainpages,hypertexnames=false,plainpages=false]{hyperref}
\hypersetup{urlcolor=blue, citecolor=blue, linkcolor=blue}

\usepackage[utf8]{inputenc}
\usepackage{amsmath}
\usepackage{amsthm}
\usepackage{amssymb}
\usepackage{amsfonts}
\usepackage{enumerate}
\usepackage{mathtools}
\usepackage{tikz-cd}
\usepackage{hhline}
\usepackage{stmaryrd}
\usepackage{enumitem}
\usepackage{centernot}
\usepackage[textwidth=25mm]{todonotes}
\setlength{\marginparwidth}{25mm}% making todonotes fit the page
\usepackage{mathrsfs}
\usepackage{comment}
\usepackage{listings}
\usepackage{multicol}
\usepackage[capitalise, noabbrev]{cleveref} %for \cref
\usepackage[noend]{algorithmic} % for the algorithmic environment
\renewcommand{\algorithmicrequire}{\textbf{Input:}}
\renewcommand{\algorithmicensure}{\textbf{Output:}}
\renewcommand\algorithmiccomment[1]{#1}
\usepackage{mathdots}
\usepackage{cleveref} % for \cref
\usepackage{nicematrix} % for NiceMatrix
\usepackage{mathalpha} % for various math fonts
\DeclareMathAlphabet{\dutchcal}{U}{dutchcal}{m}{n}
\newcommand{\dcu}{\dutchcal{u}}
\newcommand{\dcv}{\dutchcal{v}}
\newcommand{\dcw}{\dutchcal{w}}

\usepackage{array}
\newcolumntype{L}[1]{>{\raggedright\let\newline\\\arraybackslash\hspace{0pt}}m{#1}}
\newcolumntype{C}[1]{>{\centering\let\newline\\\arraybackslash\hspace{0pt}}m{#1}}
\newcolumntype{R}[1]{>{\raggedleft\let\newline\\\arraybackslash\hspace{0pt}}m{#1}}


\newcommand{\yue}[1]{{\textcolor{red!50!black}{Yue: #1}}}

\newcommand*\angles[1]{\langle #1 \rangle}

\newcommand{\intraequational}{intra-equational}
\newcommand{\intramonomial}{intra-monomial}


%%%%%%%%%%%%%%%%%%%%%%%%
%%%% Page settings %%%%%
%%%%%%%%%%%%%%%%%%%%%%%%
%\setlength{\parindent}{0pt}
%\setlength{\parskip}{6pt}
\renewcommand{\baselinestretch}{1.1}
\setlength{\oddsidemargin}{-1in} % Linker Rand 30mm vom Papierrand
\addtolength{\oddsidemargin}{30mm}
\setlength{\evensidemargin}{\oddsidemargin}
\setlength{\textwidth}{150mm} % Rechter Rand 30mm vom Papierrand

%%%%%%%%%%%%%%%%%%%%%%%%%%%%%%%%%
%%%% Begin Graphics Section %%%%%
%%%%%%%%%%%%%%%%%%%%%%%%%%%%%%%%%
\usepackage{tikz}
\usetikzlibrary{arrows}
\usetikzlibrary{decorations.pathreplacing}
\usetikzlibrary{matrix}
\usetikzlibrary{calc}
\usetikzlibrary{shapes}
\usetikzlibrary{patterns}
\usetikzlibrary{fit,backgrounds,scopes}
%%%%%%%%%%%%%%%%%%%%%%%%%%%%%%%%%
%%%% End Graphics Section %%%%%
%%%%%%%%%%%%%%%%%%%%%%%%%%%%%%%%%

%%%%%%%%%%%%%%%%%%%%%%%%%%%%%%%%%
%%%% Begin Theorem Section  %%%%%
%%%%%%%%%%%%%%%%%%%%%%%%%%%%%%%%%
\newtheoremstyle{theoremstyle}
{10pt}      %  Space above
{5pt}       %  Space below
{\itshape}  %  Body font
{}          %  Indent amount (empty = no indent, \parindent = para indent)
{\bfseries} %  Thm head font
{}         %  Punctuation after thm head
{ }      %  Space after thm head: " " = normal interword space;
% \newline = linebreak
{}          %  Thm head spec (can be left empty, meaning `normal')

\newtheoremstyle{algorithmstyle}
{10pt}      %  Space above
{5pt}       %  Space below
{}  %  Body font
{}          %  Indent amount (empty = no indent, \parindent = para indent)
{\bfseries} %  Thm head font
{}         %  Punctuation after thm head
{ }      %  Space after thm head: " " = normal interword space;
% \newline = linebreak
{}          %  Thm head spec (can be left empty, meaning `normal')

\newtheoremstyle{examplestyle}
{10pt}      %  Space above
{5pt}       %  Space below
{}          %  Body font
{}          %  Indent amount (empty = no indent, \parindent = para indent)
{\bfseries} %  Thm head font
{}         %  Punctuation after thm head
{ }      %  Space after thm head: " " = normal interword space;
% \newline = linebreak
{}          %  Thm head spec (can be left empty, meaning `normal')

\makeatletter
\newtheorem*{rep@theorem}{\rep@title}
\newcommand{\newreptheorem}[2]{%
\newenvironment{rep#1}[1]{%
 \def\rep@title{#2 \ref{##1}}%
 \begin{rep@theorem}}%
 {\end{rep@theorem}}}
\makeatother

\makeatletter % subalign = substack + align
\newcommand{\subalign}[1]{%
  \vcenter{%
    \Let@ \restore@math@cr \default@tag
    \baselineskip\fontdimen10 \scriptfont\tw@
    \advance\baselineskip\fontdimen12 \scriptfont\tw@
    \lineskip\thr@@\fontdimen8 \scriptfont\thr@@
    \lineskiplimit\lineskip
    \ialign{\hfil$\m@th\scriptstyle##$&$\m@th\scriptstyle{}##$\hfil\crcr
      #1\crcr
    }%
  }%
}
\makeatother

\newreptheorem{theorem}{Theorem}
\newreptheorem{corollary}{Corollary}
\newreptheorem{proposition}{Proposition}



\theoremstyle{theoremstyle}
\newtheorem{theorem}{Theorem}[section]
\newtheorem{lemma}[theorem]{Lemma}
\newtheorem{proposition}[theorem]{Proposition}
\newtheorem{corollary}[theorem]{Corollary}
\theoremstyle{examplestyle}
\newtheorem{example}[theorem]{Example}
\newtheorem{definition}[theorem]{Definition}
\newtheorem{definitionnotation}[theorem]{Definition and Notation}
\newtheorem{notation}[theorem]{Notation}
\newtheorem*{notation*}{Notation}
\newtheorem{recollection}[theorem]{Recollection}
\newtheorem{remark}[theorem]{Remark}
\newtheorem{assumption}[theorem]{Assumption}
\newtheorem{convention}[theorem]{Convention}
\newtheorem{construction}[theorem]{Construction}
\newtheorem{timings}[theorem]{Timings}
\theoremstyle{algorithmstyle}
\newtheorem{algorithm}[theorem]{Algorithm}
%%%%%%%%%%%%%%%%%%%%%%%%%%%%%%%%%
%%%% End Theorem Section  %%%%%
%%%%%%%%%%%%%%%%%%%%%%%%%%%%%%%%%


%%%%%%%%%%%%%%%%%%%%%%%%%%%%%%%%%%%%
%%%% Begin Newcommand Section  %%%%%
%%%%%%%%%%%%%%%%%%%%%%%%%%%%%%%%%%%%
\newcommand{\NN}{\mathbb{N}}
\newcommand{\CC}{\mathbb{C}}
\newcommand{\RR}{\mathbb{R}}
\newcommand{\QQ}{\mathbb{Q}}
\newcommand{\ZZ}{\mathbb{Z}}
\newcommand{\FF}{\mathbb{F}}
\newcommand{\TT}{\mathbb{T}}
\newcommand{\suchthat}{\;\ifnum\currentgrouptype=16 \middle\fi|\;}
\newcommand{\bigmid}{\left.\vphantom{\Big\{} \suchthat \vphantom{\Big\}}\right.}
\newcommand{\bigslant}[2]{{\raisebox{.2em}{$#1$}\left/\raisebox{-.2em}{$#2$}\right.}}

\newcommand{\lin}{{\mathrm{lin}}}
\newcommand{\nlin}{{\mathrm{nlin}}}
\newcommand{\start}{{\mathrm{start}}}
\newcommand{\current}{{\mathrm{current}}}
\newcommand{\target}{{\mathrm{target}}}


%% Math operators:
\DeclareMathOperator{\codim}{codim}
\DeclareMathOperator{\conv}{conv}
\DeclareMathOperator{\Dr}{Dr}
\DeclareMathOperator{\minors}{minors}
\DeclareMathOperator{\mult}{mult}
\DeclareMathOperator{\Newt}{Newt}
\DeclareMathOperator{\MV}{MV}
\DeclareMathOperator{\Span}{Span}
\DeclareMathOperator{\Supp}{Supp}
\DeclareMathOperator{\trop}{trop}
\DeclareMathOperator{\Trop}{Trop}
\DeclareMathOperator{\val}{val}

%%%%%%%%%%%%%%%%%%%%%%%%%%%%%%%%%%
%%%% End Newcommand Section  %%%%%
%%%%%%%%%%%%%%%%%%%%%%%%%%%%%%%%%%

\setcounter{biburllcpenalty}{7000} % penalizing overlong urls in bibliography
\setcounter{biburlucpenalty}{8000}
\raggedbottom
\begin{document}

\title[Tropical homotopy continuation on tropical linear spaces]{Tropical homotopy continuation\\ on tropical linear spaces}
\author{Oliver Daisey}
\address{Department of Mathematics, Durham University, United Kingdom.}
\email{oliver.j.daisey@durham.ac.uk}
\urladdr{https://www.durham.ac.uk/staff/oliver-j-daisey/}
\author{Yue Ren}
\address{Department of Mathematics, Durham University, United Kingdom.}
\email{yue.ren2@durham.ac.uk}
\urladdr{https://yueren.de}

\subjclass[2020]{14T10,14T90}

\date{\today}

\keywords{tropical geometry.}

\begin{abstract}
  lorem ipsum
\end{abstract}

\maketitle

\section{Introduction}

In \cite{Jensen2016arXiv}, Jensen develops a tropical analogue of homotopy continuation to compute the stable intersection of $n$ tropical hypersurfaces in $\RR^n$.  Jensen's main motivation was the computation of mixed volumes and applications to polyhedral homotopies \cite{HuberSturmfels1995}.  The algorithm was initially implemented in \textsc{gfan} \cite{gfan,Jensen2016ICMS}, where it is still powering the mixed volume computation to date.  Since then, polynomial system solvers like \textsc{HomotopyContinuation.jl} \cite{HomotopyContinuation.jl} have picked up the algorithms and are using it for polyhedral homotopies.

In this article, we will generalize Jensen's tropical homotopies to computing the stable intersection of $k$ tropical hypersurfaces and a tropical linear space of dimension $n-k$ in $\RR^n$.  Similar to Jensen, our motivation also comes from polynomial system solving.

%%% Local Variables:
%%% mode: latex
%%% TeX-master: "main"
%%% End:


\section{Background}
In this section, we will fix some notation by briefly going over some basic concepts in tropical geometry that are of immediate interest to us. These are
\begin{enumerate}
\item tropical hypersurfaces and Newton subdivisions \cite[\S 3.1]{MaclaganSturmfels2015} \cite[\S 1]{Joswig2021},
\item tropical linear spaces and Matroid subdivisions \cite[\S 4.4]{MaclaganSturmfels2015} \cite[\S 10]{Joswig2021},
\item stable intersections and mixed cells \cite[\S 3.6+4.6]{MaclaganSturmfels2015}.
\end{enumerate}

For the sake of simplicity, and unlike \cite{MaclaganSturmfels2015,Joswig2021}, we will consider tropical varieties as balanced polyhedral complexes rather than supports thereof.  This is because the multiplicity plays a major role in our paper and not having to keep track of separate multiplicity functions streamlines the notation. For the sake of brevity, we will moreover introduce all tropical varieties via their dual pictures as it is the main description we will be working with.

\begin{convention}
  For the remainder of the article, let $\TT\coloneqq(\RR\cup\{\infty\},\oplus,\odot)$ denote the min-plus tropical semiring and fix a multivariate (Laurent) polynomial ring $\TT[x^\pm]\coloneqq \TT[x_1^\pm,\dots,x_n^\pm]$.
\end{convention}

\subsection{Tropical hypersurfaces}

\begin{definition}
  Let $f=\sum_{\alpha\in\ZZ^n}c_\alpha x^\alpha\in\TT[x^\pm]$ be a tropical Laurent polynomial.\linebreak  The \emph{support} of $f$ is the set of all exponent vectors with tropically non-zero coefficient, i.e., $\mathrm{Supp}(f)\coloneqq\{\alpha\in\ZZ^n\mid c_\alpha\neq \infty\}$.  The \emph{Newton polytope} of $f$ is the convex hull of its support, and the \emph{Newton subdivision} $\Delta(f)$ is the regular subdivision on $\Supp(f)$ induced by the coefficients.

  For the sake of simplicity, we will not distinguish between tropical coefficient vectors $f'=(c_\alpha)_{\alpha\in\Supp(f)}\in \TT^{\Supp(f)}$ and tropical Laurent polynomials $f'=\sum_{\alpha\in\Supp(f)}c_\alpha x^\alpha\in\TT[x^\pm]$ with $\Supp(f')\subseteq\Supp(f)$.
\end{definition}

\begin{definition}
  Let $f=\sum_{\alpha\in\ZZ^n}c_\alpha x^\alpha\in\TT[x^\pm]$ be a tropical Laurent polynomial. Each cell of the Newton subdivision $\delta\in\Delta(f)$ defines a dual polyhedron as follows:
  \[ C_\delta(f)\coloneqq \{ w\in\RR^n\mid \text{the minimum in } f(w) \text{ is attained at the vertices of } \delta \}. \]
  The \emph{tropical hypersurface} of $f$ is the set of all such dual polyhedra arising from positive-dimensional cells of the Newton subdivision,
  \[ \Trop(f)\coloneqq \{C_\delta(f)\mid \delta\in\Delta(f), \dim(\delta)>0 \}. \]
  Moreover, maximal $C_\delta(f)\in\Trop(f)$ are dual to one-dimensional $\delta\in\Delta(f)$.  The multiplicity of each maximal $C_\delta(f)\in\Trop(f)$ is the lattice length of $\delta$.
\end{definition}

\begin{example}

\end{example}

\subsection{Tropical linear spaces}

\begin{definition}
  Let $k,n\in\ZZ_{\geq 0}$, $k\leq n$.  A \emph{tropical Pl\"ucker vector} is a point $p\in \TT^{\binom{k}{n}}$ such that for all $\sigma\in\binom{[n]}{k-1}$ and $\tau\in\binom{[n]}{k+1}$ with $\sigma\subseteq\tau$ the following minimum is attained at least twice:
  \[ \bigoplus_{i\in\tau} p_{\sigma\cup\{i\}}\odot p_{\tau\setminus\{i\}}. \]
  The \emph{support} of a tropical Pl\"ucker vector $p$ are the indices of its tropically non-zero entries, i.e., $\Supp(p)\coloneqq\{ B\in S\mid p_B\neq\infty\}$. The \emph{Dressian} $\Dr(\Supp(p))$ is the set of tropical Pl\"ucker vector with support $\Supp(p)$.


  To each tropical Pl\"ucker vector $p$ we can assign a tropical Laurent polynomial $f_p\coloneqq \bigoplus_{B\in \Supp(p)} p_B\odot x^{e_B}\in\TT[x^\pm]$, where $e_B\in\{0,1\}^n$ is the indicator vector of $B$.  The \emph{matroid polytope} of $p$ is the Newton polytope of $f_p$, and the \emph{matroid subdivision} of $p$ is the Newton subdivision $\Delta(f_p)$.

\end{definition}

\begin{definition}
  A cell of the matroid subdivision $\delta\in\Delta(f_p)$ is \emph{loopless} if $\bigcup_{e_B\in\delta}B=[n]$, and each cell defines a dual polyhedron as follows:
  \[ C_\delta(f_p)\coloneqq \{ w\in\RR^n\mid \text{the minimum in } f_p(w) \text{ is attained at the vertices of } \delta \}. \]
  The \emph{tropical linear space} of $p$ is the set of all such dual polyhedra arising from loopless cells of the Matroid subdivision,
  \[ L_p\coloneqq \{C_\delta(f_p)\mid \delta\in\Delta(f_p), \delta \text{ loopless} \}. \]
  Each maximal cell $C_\delta(f)\in L_p$ has weight $1$.
\end{definition}

\begin{example}

\end{example}

\subsection{Stable intersections}

\begin{definition}
  Define stable intersection two ways:
  \begin{enumerate}
  \item Without perturbation
  \item With perturbation
  \end{enumerate}
\end{definition}

\begin{definition}
  (Generalised) mixed cells and mixed cell candidates
  \begin{enumerate}
      \item A mixed cell candidate is $s=(s_p,s_1,\dots,s_k)\subseteq \Supp(f_p)\times\Supp(f_1)\times\dots\times\Supp(f_k)$ with $s_p$ loopless and $|s_i|=2$.
  \end{enumerate}
\end{definition}

\begin{definition}
  Let $p\in \TT^{\binom{n}{k}}$ be a tropical Pl\"ucker vector and let $f=(f_1,\dots,f_k)\in\TT[x^\pm]^k$ be tuple of tropical polynomials.  In \cref{sec:algorithm}, these will be the starting data of our tropical homotopies.

  For any mixed cell candidate $s=(s_p,s_1,\dots,s_k)\subseteq \Supp(f_p)\times\Supp(f_1)\times\dots\times\Supp(f_k)$, we define the (generalised) mixed cell cone to be
  \begin{align*}
    C_s\coloneqq &\Big\{ (f_p',f_1',\dots,f_n')\in \RR^{\Supp(f_p)}\times\RR^{\Supp(f_1)}\times\dots\times\RR^{\Supp(f_k)} \bigmid \\
    &\hspace{7mm}\conv(s_p)+\conv(s_1)+\dots+\conv(s_k) \text{ is a face of }\Delta(f_p'\odot f_1'\odot\dots\odot f_k')\Big\}.
  \end{align*}

  Given a cone $C\subseteq \RR^{\Supp(f_p)}\times\RR^{\Supp(f_1)}\times\dots\times\RR^{\Supp(f_k)}$, we say a tropical point $(p',f_1',\dots,f_n')\in\TT^{\Supp(p)}\times\TT^{\Supp(f_1)}\times\dots\times\TT^{\Supp(f_k)}$ is in the closure of $C$, or $(p',f_1',\dots,f_n')\in\overline C$, if there is an unbounded path in $C$ converging to it.
\end{definition}
%%% Local Variables:
%%% mode: latex
%%% TeX-master: "main"
%%% End:


\section{The running example}
Run through an illustrative example involving:
\begin{enumerate}
\item two-codimensional tropical linear space in $\RR^4$,
\item one (homogeneous) tropical hypersurface in $\RR^4$.
\end{enumerate}

Both polyhedral complexes are invariant under translation by $\RR\cdot(1,1,1,1)$, as will be their stable intersection.  We can thus draw three-dimensional pictures by restricting the tropical ambient space to $\RR^3\cong \{0\}\times\RR^3$ and the Newton ambient space to $\{w+x+y+z=d\}$ for a suitable $d$.

\subsection*{start}
$f_{\start} = w^2\oplus wx\oplus wy\oplus wz\oplus x^2\oplus xy\oplus xz\oplus y^2\oplus yz\oplus z^2$,
$p_{\start} = 0\in\RR^{\binom{4}{2}}$

\subsection*{target}
$f_{\target} = x_0x_1\oplus x_0x_3\oplus (-1)\odot x_1x_3$,
$p_{\target} = (0,0,\infty,0,0,0)\in\RR^{\binom{4}{2}}$ with $p_{\target,\{03\}}=\infty$.

\subsection*{expectation}
\begin{enumerate}
\item $p_0$ and $f_0$ have a single mixed cell of volume $2$.
\item Fix $p$ at $p_0$, change $f$ from $f_0$ to $f_1$: one mixed cell of volume $2$ breaks into two mixed cells of volume $1$.
\item Change $p$ from $p_0$ to $p_1$, fix $f$ at $f_1$: one mixed cell of volume $1$ diverges.
\end{enumerate}

%%% Local Variables:
%%% mode: latex
%%% TeX-master: "main"
%%% End:


\section{The algorithm}\label{sec:algorithm}
In this section, we go over the actual algorithm. \cref{subsec:homotopyContinuation} contains the top-level algorithm, while \cref{subsec:startingSystem} describes our starting systems, and \cref{subsec:multistellarFlips} describes how the data is updated along the way.

\subsection{Tropical homotopy continuation on tropical linear spaces}\label{subsec:homotopyContinuation}

\begin{algorithm}\
  \begin{algorithmic}[1]
    \REQUIRE{$(p,f)$, where
      \begin{enumerate}[leftmargin=*]
      \item $p\in\TT^{\binom{n}{k}}$ a tropical Pl\"ucker vector,
      \item $f=(f_1,\dots,f_k)\in\TT[x]^k$ a tuple of tropical polynomials.
      \end{enumerate}
    }
    \ENSURE{Mixed cells dual to the stable intersection $L_p\wedge\Trop(f_1)\wedge\!\cdots\!\wedge\Trop(f_k)$.}
    \STATE Construct starting data
    \[ (p_\start,f_\start,\mathcal S)\coloneqq\mathtt{StartingData}(p,f). \vspace{-1em} \]
    \STATE Construct the starting mixed cone cells $\mathcal C\coloneqq\{C_s\mid s\in\mathcal S\}$,
    and a tropical path
    \[ h\colon [0,\infty]\rightarrow \Dr(\Supp(p_\start))\times \TT^{\Supp(f_{\start,1})}\times\dots\times \TT^{\Supp(f_{\start,k})} \]
    with $h(0)\sim(p_\start,f_\start)$ and $h(\infty)\sim(p,f)$, where $a\sim b$ if they induce the same subdivisions on the relevant supports.
    \STATE Set $t=0$.
    \WHILE{$(p,f)\notin\overline{\bigcap_{C_s\in \mathcal C}C_s}$}
    \STATE Identify the set $\mathcal C'\subseteq\mathcal C$ of mixed cell cones who are first breached by $h|_{[t,\infty]}$.
    \STATE Update $\mathcal C$ with the adjacent mixed cell cones:
    \[ \mathcal C\coloneqq \mathcal C\setminus \mathcal C' \cup \bigcup_{C_s\in\mathcal C'} \mathtt{MultistellarFlip}(C_s,h|_{[t,\infty]}). \]
    \STATE Pick any $t\in h([t,\infty])\cap \bigcap_{C_s\in \mathcal C}C_s$.
    \ENDWHILE
    \RETURN{$\{s\mid C_s\in \mathcal C\}$.}
  \end{algorithmic}
\end{algorithm}

\subsection{Starting system for total degree homotopies}\label{subsec:startingSystem}

\begin{algorithm}[total degree starting data]\
  \begin{algorithmic}[1]
    \REQUIRE{$(p,f)$, where
      \begin{enumerate}[leftmargin=*]
      \item $p\in\TT^{\binom{n+1}{k+1}}$ a tropical Pl\"ucker vector,
      \item $f=(f_1,\dots,f_k)\in\TT[x]^k$ a tuple of homogeneous tropical polynomials.
      \end{enumerate}
    }
    \ENSURE{$(p_\start,f_\start,\mathcal S)$ starting data for tropical homotopies, i.e.,
      \begin{enumerate}
      \item $p_\start\in\TT^{\binom{n+1}{k+1}}$ tropical Pl\"ucker vector with $\Supp(p)\subseteq \Supp(p_\start)$,
      \item $f_\start=(f_{\start,1},\dots,f_{\start,k})\in\TT[x^\pm]^k$ with $\Supp(f_i)\subseteq\Supp(f_{\start,i})$,
      \item $\mathcal S\subseteq\Delta(f_{p_\start}\odot f_{\start,1}\odot\cdots\odot f_{\start,k})$ mixed cells dual to the transverse stable intersection  $L_{p_\start}\wedge\Trop(f_{\start,1})\wedge\cdots\wedge\Trop(f_{\start,k})$.
      \end{enumerate}
    }
    \STATE Set $p_\start\coloneqq 0 \in\TT^{\binom{n+1}{k+1}}$.
    \STATE Let $\ell_1,\dots \ell_k\in\TT[x^\pm]$ be of the form
    \begin{equation*}
      \ell_i \coloneqq \bigoplus_{j=0}^n (-c_{i,j})\odot x_j \qquad\text{where } c_{i,j}=
      \begin{cases}
        -1 & \text{if }j>k,\\
        +1 & \text{if }j=i,\\
        0 & \text{otherwise}.
      \end{cases}
    \end{equation*}
    \STATE Set $f_\start\coloneqq(f_{\start,1},\dots,f_{\start,k})$ with $f_{\start,i}=(\ell_i)^{d_i}$ where $d_i \coloneqq \deg(f_i)$.
    \STATE Set $\mathcal S\coloneqq\{ \sigma_p+\sigma_1+\dots+\sigma_k \}$ where $\sigma_p=\conv(e_0,e_{k+1}, \dots, e_{n})$ and $\sigma_i=d_i\cdot\conv(e_0, e_i)$.
    \RETURN{$(p_\start,f_\start,\mathcal S)$.}
  \end{algorithmic}
\end{algorithm}
\begin{proof}
  By construction, we have $\Supp(p)\subseteq \Supp(p_\start)$ and $\Supp(f_i)\subseteq\Supp(f_{\start,i})$. It remains to be shown that $\sigma_p+\sigma_1+\dots+\sigma_k$ is the only mixed cell,

  First, we show that the point $w\coloneqq \sum_{i=1}^ke_i\in\RR^{n+1}$ is contained in the intersection $|L_{p_\start}|\cap|\Trop(f_{\start,1})|\cap\cdots\cap|\Trop(f_{\start,k})|$, which coincides with the intersection $|L_{p_\start}|\cap|\Trop(\ell_{1})|\cap\cdots\cap|\Trop(\ell_{k})|$.  Note that $L_{p_\start}$ is the Bergman fan of the uniform matroid of rank $k+1$ and that $\Trop(\ell_i)$ is the standard tropical hyperplane translated by $c_i$, i.e.,
  \begin{align*}
    L_{p_\start} &\supseteq \Big\{ \underbrace{\sum_{i\in I}\RR_{\geq 0} \cdot e_i + \RR \cdot (1, \dots, 1) }_{\eqqcolon\tau_I}\bigmid I\in\textstyle\binom{[n+1]}{k} \Big\}\quad \text{and}\\
    \Trop(\ell_{i})&\supseteq \Big\{ \underbrace{c_i+\sum_{k\neq j_1,j_2} \RR_{\geq 0} \cdot e_k + \RR \cdot (1, \dots, 1)}_{\eqqcolon\tau_{i,j_1j_2}} \mid \{j_1,j_2\}\in \binom{[n+1]}{2} \Big\}.
  \end{align*}
  In particular, $w=\sum_{i=0}^ke_i\in\Relint(\tau_{\{0\dots k\}})$ and $w=c_i+\sum_{j\neq i}e_i\in\Relint(\tau_{i,i})$.  Hence $w \in |L_{p_\start}|\cap|\Trop(\ell_{1})|\cap\cdots\cap|\Trop(\ell_{k})|$.  Moreover, $w$ lies in the interior of the maximal polyhedra of the intersects.

  Next, we show that $w$ is the only point in the intersection.  As all intersects are tropical linear spaces, it suffices to show that the $L_{p_\start}$ and the $\Trop(\ell_i)$ intersect transversally around $w$.  This can be seen from the normal vectors of the maximal cells that contain $w$:
  \begin{align*}
    \Span(\tau_{\{1\dots k\}}) &= \Span(e_0-e_{k+1},\dots,e_0-e_{n})^\perp \quad\text{and}\\
    \Span(\tau_{i,i}-c_i) &= \Span(e_0-e_i)^\perp \text{ for }i=1,\dots,k.
  \end{align*}
  The fact that that all normals are linearly independent shows that the intersection is transversal.

  The fact that $\sigma_p+\sigma_1+\dots+\sigma_k$ is the only mixed cell now follows from the fact that $\sigma_p$ is dual to $\tau_{\{1\dots k\}}$ and that $\sigma_i$ is dual to $\tau_{i,i}$.
\end{proof}

% \subsection{Homotopy paths for linearly realisable targets}
% \begin{algorithm}\
%   \label{alg:homotopy paths}
%   \begin{algorithmic}[1]
%     \REQUIRE{$(A_\start,A_\target)$, where $A_\start,A_\target\in K^{(k+1)\times (n+1)}$ for a field $K$ with valuation $\val\colon K\rightarrow\TT$ such that
%       \begin{enumerate}
%       \item $\val(\minors_{(k+1)\times(k+1)}(A_\start))=p_\start$ and
%       \item $\val(\minors_{(k+1)\times(k+1)}(A_\target))=p_\target$.
%       \end{enumerate}
%     }
%     \ENSURE{$h_p\in\TT[t]^{\binom{n+1}{k+1}}$ a tropical path such that $h_p([0,\infty])\in\Dr(\Supp(p_\start))$ with
%       \begin{enumerate}
%       \item $h_p(0)\sim p_\start$ and
%       \item $h_p(\infty)\sim p_\target$.
%       \end{enumerate}
%     }
%     \STATE Set $h_p(s)\coloneqq \trop(\minors_{(k+1)\times(k+1)}(s\cdot A_\start + A_\target))\in\TT[s]^{\binom{n+1}{k+1}}$, where $\trop(\cdot)$ denotes coordinate-wise tropicalization of polynomials in $K[s]$ to polynomials in $\TT[s]$ by applying coefficient-wise $\val(\cdot)$.
%     \STATE Compute the minimum of all tropical roots of the entries of $h_p$, i.e., compute $s_0\coloneqq \bigoplus_{i=1,\dots,\binom{n+1}{k+1}} \bigoplus_{r\in V(h_{p,i})} r$, where $V(\cdot)$ stands for the tropical roots of a tropical polynomial.
%     \RETURN{$h_p(s/s_0)$.}
%   \end{algorithmic}
% \end{algorithm}

\subsection{Multistellar flips}\label{subsec:multistellarFlips}

%%% Local Variables:
%%% mode: latex
%%% TeX-master: "main"
%%% End:


\renewcommand*{\bibfont}{\small}
\printbibliography

\end{document}
%%% Local Variables:
%%% mode: latex
%%% TeX-master: t
%%% End:
